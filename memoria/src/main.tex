\documentclass[a4paper,10pt]{article}

\usepackage{url}

\usepackage[T1]{fontenc}
\usepackage[utf8]{inputenc}
\usepackage[spanish, es-tabla]{babel}

\clubpenalty=10000
\widowpenalty=10000

\usepackage[gen]{eurosym}

\usepackage{xcolor}
\usepackage{setspace}
\usepackage{float}
\usepackage{biblatex}
\addbibresource{fuentes.bib}
\setlength{\parindent}{0cm}
\usepackage{amsmath, amssymb}
\usepackage{svg}

% Listados de código
\usepackage{minted}
\usepackage{caption}
\newenvironment{code}{\captionsetup{type=listing}}{}
% \SetupFloatingEnvironment{listing}{name=Source Code}
\renewcommand{\listingscaption}{Código}
\BeforeBeginEnvironment{minted}{\vspace{-.3cm}}
\AfterEndEnvironment{minted}{\vspace{-.3cm}}

\usepackage{csquotes}

\renewcommand{\it}[1]{\textit{#1}}
\renewcommand{\bf}[1]{\textbf{#1}}
\renewcommand{\tt}[1]{\texttt{#1}}

% Define el comando para introducir código
 \newenvironment{codigo}[1]
 {
    \captionsetup{type=listing}
    \VerbatimEnvironment
    \begin{minted}[fontsize=\normalsize, xleftmargin=0pt, xrightmargin=0pt, frame=lines]{#1}}
 {\end{minted}}

\usepackage{booktabs} % Allows the use of \toprule, \midrule and \bottomrule in tables for horizontal lines
\usepackage{graphicx}
\graphicspath{{imagenes/}{../imagenes/}} % Declara dos paths, uno con respecto a este documento y otro respecto a la carpeta secciones

\parskip=0.4cm % Distancia entre párrafos

\usepackage[margin=3cm, headheight=17.9pt]{geometry}
 \usepackage{fancyhdr}
\fancyhf{}
\lhead{\includegraphics[width=0.6cm]{Log4Shell_logo.png}}
\fancyhead[C]{\emph{Seguridad informática -- 2021-22}}
\rhead{\emph{Trabajo de asignatura}}
\fancyfoot[C]{\thepage}

\usepackage{subfiles} % Permite incluir ficheros .tex en el texto

\usepackage{hyperref} % Hace que el índice sean hiperenlaces al documento


\begin{document}

\begin{titlepage}
    \centering
    \includegraphics[width=1 cm]{logoUZ.jpg}
    
    \textsc{\large Universidad de Zaragoza}
    \rule{\textwidth}{1.6pt}\vspace*{-\baselineskip}\vspace*{2pt} % Thick horizontal rule
    \rule{\textwidth}{0.4pt} % Thin horizontal rule
    
    \vfill
    
    {\LARGE \scshape Seguridad informática}
                
    \vspace{2cm}            

    {\bfseries \Huge Memoria del trabajo de asignatura}
    
    \vspace{.5cm} 
    
    {\bfseries \Large Estudio de la vulnerabilidad Log4Shell}
    
    \vspace{3cm}    
    
   

    {\scshape Grupo inf5\_A 1:}


    \vspace{0.2cm}
    
    \large
    \begin{tabular}{c l l}
    \large             & Adrián Martín Marcos       & 756524 \\
    \large             & Pablo López-Alonso Alonso       &  759836\\
     \end{tabular}

    \vfill
    
    \large{Zaragoza, España}
    
    {Curso 2020\,--\,2021}

    \vfill

    \includegraphics[width=5.0cm]{EINA.png}
   
\end{titlepage}

\pagenumbering{Roman}


\vspace*{2cm}
\section*{\hfil Introducción \hfil}
\addcontentsline{toc}{section}{Introducción}

En este trabajo se va a estudiar la vulnerabilidad \it{Log4Shell}\cite{cve-log4shell} del la biblioteca de logging \it{Log4J} para Java, haciendo especial hincapié en las posibilidades de su explotación.

% Esto es lo que dijo Simona en el correo (resume muy bien los apartados)
La idea es estudiar la vulnerabilidad, entendiendo qué principios de desarrollo software no se han tenido en cuenta, cómo se puede explotar y las posibles contramedidas que se han tomado. Además de ello, se ha conseguido hacer ataques explotándola usando máquinas virtuales de la asignatura.

En los diferentes apartado se tratará de seguir el mismo esquema que en la presentación de la vulnerabilidad \it{shellshock}. De igual forma, los objetivos marcados para este trabajo serán los mismos que en la práctica relacionada con \it{shellshock} (entender la vulnerabilidad y explotarla para obtener un \it{reverse shell}).

\pagebreak

\begin{spacing}{0.1}
\tableofcontents
\end{spacing}
%\listoffigures
%\listoftables

\pagebreak

\setcounter{page}{1}
\pagenumbering{arabic}

\pagestyle{fancy}

\subfile{secciones/1-Log4j}

\clearpage{}
\newpage

\subfile{secciones/2-Log4Shell}

\clearpage{}
\newpage

\subfile{secciones/3-Exploit}

\clearpage{}
\newpage

\subfile{secciones/4-ReverseShell}

\clearpage{}
\newpage

\subfile{secciones/5-Mitigation}

\clearpage{}
\newpage

\nocite{*}
\addcontentsline{toc}{section}{Referencias} % Añade en índice
\printbibliography

\end{document}


